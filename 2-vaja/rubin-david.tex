% To je predloga za poročila o domačih nalogah pri predmetih, katerih
% nosilec je Blaž Zupan. Seveda lahko tudi dodaš kakšen nov, zanimiv
% in uporaben element, ki ga v tej predlogi (še) ni. Več o LaTeX-u izveš na
% spletu, na primer na http://tobi.oetiker.ch/lshort/lshort.pdf.
%
% To predlogo lahko spremeniš v PDF dokument s pomočjo programa
% pdflatex, ki je del standardne instalacije LaTeX programov.

\documentclass[a4paper,11pt]{article}
\usepackage{a4wide}
\usepackage{fullpage}
\usepackage[utf8x]{inputenc}
\usepackage[slovene]{babel}
\selectlanguage{slovene}
\usepackage[toc,page]{appendix}
\usepackage[pdftex]{graphicx} % za slike
\usepackage{setspace}
\usepackage{color}
\definecolor{light-gray}{gray}{0.95}
\usepackage{listings} % za vključevanje kode
\usepackage{hyperref}
\renewcommand{\baselinestretch}{1.2} % za boljšo berljivost večji razmak
\renewcommand{\appendixpagename}{Priloge}

\lstset{
  basicstyle=\ttfamily,
  breaklines=true,
  postbreak=\mbox{\textcolor{red}{$\hookrightarrow$}\space},
}

\lstset{ % nastavitve za izpis kode, sem lahko tudi kaj dodaš/spremeniš
language=Python,
basicstyle=\footnotesize,
basicstyle=\ttfamily\footnotesize\setstretch{1},
backgroundcolor=\color{light-gray},
}

\title{%
  Vaja 2\\
  \large Postavitev in upravljanje računalniških oblakov}
\author{David Rubin \\ (david.rubin@student.um.si)}
\date{\today}

\begin{document}

\maketitle

\section{Opis naloge}

CIlj naloge je bila mreža virtualnih računalnikov, ki imajo med sabo omogočeno SSH povezavo. Torej s pomočjo VirtualBox smo ustvarili gručo 10 računalnikov, na katere se lahko povežemo s protokolom SSH. Za demonstracijo delovanja smo uporabili še program \textit{parallel-ssh}, ki omogoča pošiljanje ukaza na več gostiteljev.

\section{Opis rešitve}

Rešitev te naloge sem implementiral kot 2 skripti. Prva skripta, vidna v bloku~\ref{clonevm}, prikazuje del, kjer iz osnovne instance (\textit{base box}) kreiramo 9 kloniranih. V tej osnovni instanci imamo nastavljena 2 omrežna vmesnika. Eden je NAT in omogoča dostop do svetovnega spleta, drugi pa je Host Only in omogoča povezavo med virtualkami. Skripta nato s pomočjo \textit{vboxmanage} instance klonira, jih zažene, počaka, da se jim dodelijo IP naslovi (da je zagon popolen), in s temi ip naslovi zgradi veljaven SSH config datoteko, s pomočjo katere se kasneje povezujemo na računalnike.

V drugi skripti, prikazani v bloku~\ref{connect}, pa je implementirana logika za povezovanje na računalnike. Na vsako instanco se najprej kopira ssh konfiguracijska datoteka (primer je prikazan v bloku~\ref{config}), nato pa se na instanco povežemo in iz nje zaženemo ukaz \textit{parallel-ssh}. Izhod tega ukaza je prikazan v bloku~\ref{output}.

\lstinputlisting[language=bash,label=clonevm,caption=Koda programa za generiranje klonov instance]{cluster.sh}

\lstinputlisting[language=bash,label=connect,caption=Koda programa za povezovanje med instancami]{connect.sh}

\lstinputlisting[language=bash,label=config, caption=Primer konfiguracijske datoteke]{config}


\lstinputlisting[language=bash,label=output,caption=Prikaz delovanja programa za povezovanje]{machines-output.txt}


\section{Izjava o izdelavi domače naloge}
Domačo nalogo in pripadajoče programe sem izdelal sam.

\end{document}
